%% LyX 1.6.4 created this file.  For more info, see http://www.lyx.org/.
%% Do not edit unless you really know what you are doing.
\documentclass[english]{amsart}
\usepackage[T1]{fontenc}
\usepackage{array}
\usepackage{rotating}
\usepackage{amsthm}

\makeatletter

%%%%%%%%%%%%%%%%%%%%%%%%%%%%%% LyX specific LaTeX commands.
%% Because html converters don't know tabularnewline
\providecommand{\tabularnewline}{\\}

%%%%%%%%%%%%%%%%%%%%%%%%%%%%%% Textclass specific LaTeX commands.
\numberwithin{equation}{section} %% Comment out for sequentially-numbered
\numberwithin{figure}{section} %% Comment out for sequentially-numbered

\makeatother

\usepackage{babel}

\begin{document}

\title{newton in general relativty}


\date{1.8.2010}


\address{no addresss \\
somewhere \\
somewhere ZIP}


\curraddr{Home}


\email{michael\_erdmann@snafu.de}

\maketitle

\section*{http://michaelerdmann-berline.de/}


\keywords{Basic Assumptions}


\thanks{.. f..}
\begin{abstract}
The intention of this paper is to demonstrate that Newtons laws of
movement for bodies can be rewirtten in a form which is naturally
compatible with general relatity (GR) in order to trace the origin
of the mass back to its origins within the energy stress tensor of
the field equation for gravity.
\end{abstract}

\section{Introduction}

This paper follows the following program:
\begin{enumerate}
\item We will chalk out how classical physics is described in comparison
with GR
\item Rewriting Newtons statement is terms of GR
\item Guessing the Metric from the corresponding 
\item Reflection about the validty of the results
\item Some examples
\end{enumerate}

\section{Presentation of Physics in GR}

In general relativity objects are described by for values $<x^{0},x^{1},x^{2},x^{3}>$.
In classical physics, the space has not properties except for the
fact to allow the address and event witin this space using euclid
coordinates. Within the model two independant coordinates are used,
the spital coordinates $<x^{1},x^{2},x^{3}>$and the time $t$. 

\noindent %
\begin{table*}[t]
\begin{sideways}
\begin{tabular}{|>{\raggedright}p{0.07\paperwidth}|>{\raggedright}p{0.5\paperwidth}|>{\raggedright}p{0.4\paperwidth}|}
\hline 
Concept & General Relativistic & Classical Physics\tabularnewline
\hline
\hline 
Space/Particles & Space is understood as a Rieman Manifold; which means there is a unique
mapping $\varphi$ between events and the $R^{4}$. But there is no
simple way of comparing two point's $P$ and $Q$.

Time is eperienced by a moving observer based on the change of paramters
along its trajectory, which means 

\[
\frac{d}{d\lambda}(f\circ c)(\lambda)=\frac{\partial f}{\partial x^{\alpha}}\mid_{c(\lambda)}\frac{dc^{\alpha}}{d\lambda}\Rightarrow\triangle f=\frac{\partial f}{\partial x^{\alpha}}\triangle s\]




The curveture of the space is determined by

\[
R_{\mu\nu}(x)=k\left(T-\frac{1}{2}(Tg)\right)_{\mu\nu}(x)\]


The Rieman curvature tensor is related to the Cirstophel Symbols which
are determined from the metric. & The space where every thing happens in is an euclidian space but the
properties of space in terms of geometry are not changing. All points
can be transformed into each other by a simple translation and rotation.

Two independant coordinates called time $t$ and location$<x^{1},x^{2},x^{3}>$are
used to denote a certain particle. \tabularnewline
\hline 
Trajectories & $x(\lambda)=<x^{0}(\lambda),x^{1}(\lambda),x^{2}(\lambda),x^{3}(\lambda)>$.
The parameter $\lambda$is choosen in such a way that the comoving
obeserver is able to determine time. & The parameter $t$ is called time and is used to parameterise the
trajectories $x(t)=<x^{1}(\lambda),x^{2}(\lambda),x^{3}(\lambda)>$

All clocks can be synchronized in such a way that the following transformation
rule holds $t'=at+b$\tabularnewline
\hline 
Motion  & The motion of particles is determined by the geodesic equation. 

\[
\frac{d^{2}c^{\alpha}}{d\lambda^{2}}=\Gamma_{\beta\gamma}^{\alpha}\frac{dc^{\beta}}{d\lambda}\frac{dc^{\gamma}}{d\lambda}\]


The Christoffel Symbols are encapsulating the geometry of the space
where a test particle is moving in

\[
\Gamma_{\alpha\beta\gamma}=\frac{1}{2}\left(\frac{\partial g_{\beta\gamma}}{\partial x^{\alpha}}-\frac{\partial g_{\alpha\beta}}{\partial x^{\gamma}}+\frac{\partial g_{\alpha\gamma}}{\partial x^{\beta}}\right)\]


\[
\Gamma_{\beta\gamma}^{\alpha}=g^{\delta\alpha}\Gamma_{\beta\delta\gamma}\]
 & The motion of particles is only changed by imposed forces. If no force
is imposed the particle continues moving.

\raggedright{}\[
k^{i}\mid_{c(\lambda)}=m\frac{d^{2}c^{i}}{dt^{2}}\]
\tabularnewline
\hline 
Transformation &  & \tabularnewline
\end{tabular}
\end{sideways}

\caption{Classical\label{eq:}tion vs. GR notion}

\end{table*}



\section{Newton in the notion of GR}

Lets assume a curve $c(\lambda)=<t(\lambda),c^{i}(\lambda)$> with
$i\in1..3$. Newtons key statements are formulated in terms of $t$
and $c^{i}$ which should be understood as completly independant variable
which expressed by using different indicies for the spital coordinates
$i=1..3$ and by denoting time explicitly by $t$.

\begin{equation}
k^{i}\mid_{c(\lambda)}=m\frac{d^{2}c^{i}}{dt^{2}}\label{eq:sctio}\end{equation}


The equation \ref{eq:sctio} simply states the change of movement
of a mass $m$ requires a force $k$ to be imposed on the mass. This
basic statement is assumed to be valid for all inertial systems with
respect to space. With respect to time the assumption is that all
clocks in all reference system can by synchronized by the simple formuar
$t'=\alpha t+b$ . In the terms of SR, this transformatiion behaviour
is specific for low speed intertial systems.

Lets assume a curve $c(\lambda)=<t(\lambda),c^{i}\circ t(\lambda)$>
with $i\in1..3$. In this formualtion for every cooridnate system,
experienced time in any reference frame can be determined by the simple
formular $t_{2}=at_{1}+b$

\begin{equation}
\frac{d^{2}c}{d\lambda^{2}}=<\frac{d^{2}c^{0}}{d\lambda^{2}},\frac{d^{2}c^{i}}{d\lambda^{2}}>=<0,\frac{d^{2}c^{i}}{d\lambda^{2}}>\end{equation}


\begin{equation}
\frac{d^{2}}{d\lambda^{2}}(c^{i}\circ t)(\lambda)=\frac{d^{2}c^{i}}{dt^{2}}\left(\frac{dt}{d\lambda}\right)^{2}\Rightarrow\frac{d^{2}c^{i}}{d\lambda^{2}}+\frac{1}{m}k^{i}\mid_{c(\lambda)}\left(\frac{dt}{d\lambda}\right)^{2}=0\label{eq:newton}\end{equation}


Comparing equation \ref{eq:newton} with the geodesic equation \ref{eq:geo}
yields Christophel symbols $\Gamma_{\beta\gamma\delta}^{\alpha}$
which vanishes for most of the components.

\begin{equation}
\frac{d^{2}c^{\alpha}}{d\lambda^{2}}=\Gamma_{\beta\gamma}^{\alpha}\frac{dc^{\beta}}{d\lambda}\frac{dc^{\gamma}}{d\lambda}\label{eq:geo}\end{equation}


\begin{equation}
\Gamma_{00}^{i}=\begin{cases}
\frac{1}{m}k^{i}\mid_{c(\lambda)} & i=1,2,3\\
0 & else\end{cases}\end{equation}


Using the the equation for the Christophel Symbols, yields under the
assumption of a symetric metric $g_{\alpha\beta}=g_{\beta\alpha}$the
following result

\begin{equation}
\Gamma_{00}^{i}=g^{\alpha i}\Gamma_{0\alpha0}=g^{\alpha i}\frac{\partial g_{00}}{\partial x^{\alpha}}=\frac{1}{m}k^{i}\end{equation}


which can be interpreted as if an force changes the geometry of space.


\subsection{Newton law of gravity}

\begin{equation}
\frac{d^{2}c^{i}}{dt^{2}}+\frac{\partial^{2}\phi}{\partial x^{i,2}}\mid_{c(t)}=0\end{equation}



\subsection{Non Interacting Particles}

\begin{equation}
T=diag(\rho c^{2},0,0,0)\end{equation}


\begin{equation}
R_{\mu\upsilon}(x)=k\left\{ T_{\mu\nu}(x)-\frac{1}{2}(Tg)_{\mu\nu}(x)\right\} \end{equation}


\begin{equation}
R_{00}=\rho c^{2}k\left\{ 1-\frac{g_{00}}{2}\right\} \end{equation}

\begin{thebibliography}{4}
\bibitem{1}As an example; Reiner Oloff, Geometrie der Raumzeit, Vieweg,
ISBN 3-528-26917-0

\bibitem{2}\emph{Ta-Pei Chend, Relativity, Gravitation and Cosmology;
Oxford Master Series, ISBN 0-19-852957-0}.

\bibitem{3-2}American Mathematical Society, 

\bibitem{3}American Mathematical Society, 
\end{thebibliography}

\end{document}
